% Annotated Bibliography for ASEM 2405
% Drafted 04/29/2020 by Ari Dyckovsky

% Document Class Definition
\documentclass[12pt,a4paper]{article}
\usepackage[margin=0.75in]{geometry}

\usepackage[english]{babel}
\usepackage{csquotes}

% Block comments enabled
\usepackage{comment}

% Bibliography management via Biblatex and Biber
\usepackage[
  backend=biber,
  style=apa,
]{biblatex}
\addbibresource{references.bib}

% Packages

% Commands
\newcommand{\categorization}[1]{[ \textbf{ #1 } ]}

% Title, Author, and Date of document

\title{ASEM 2405: Annotated Bibliography}
\author{Ari Dyckovsky}
\date{\today}


\begin{document}

\maketitle

\section*{Categorized Annotations}

\begin{description}

  \item[\textcite{Wright1992}] Different mood states lead individuals to
subjectively judge the probability of future events. Unsuspected mood biases
might directly impact a decision and its consequences. \categorization{P}

  \item[\textcite{Hockey2000}] Both natural moods and induced moods were found to
have relatively small impacts on risky decision evaluation when compared to the
influence of fatigue. However, an interaction was also found between state
anxiety and fatigue, such that anxious individuals preferred safer options than
non-anxious individuals when fatigue levels were low or moderate. \categorization{P}

  \item[\textcite{Yuen2003}] Induced elated, neutral or depressed mood states correlated
to risky-taking tendencies during Choice Dilemmas Questionnaire. Depressed mood
less likely to take risks than those in neutral or elated mood. \categorization{P}

  \item[\textcite{Must2006}] Severely depressed individuals tend to perform worse
than non-depressed individuals on modified Iowa Gambling Task with large
punishments and large future payouts. The results are based upon medicated
individuals with diagnosed major depressive disorder, and further work needs to
qualify results for unmedicated depressed individuals. \categorization{P, N}

  \item[\textcite{VanKnippenberg2010}] Groups make decisions often without taking
full advantage of distributed information amongst group members, and affective
state may by a factor to explain this suboptimal group strategy. Positive group
mood associates to lower probabilitiees of utilizing distributed information
than neutral or negative group moods, leading to lower quality decisions. \categorization{P, E}

  \item[\textcite{VonHelversen2011}] Since previous work on the influence of
depression on decisions was inconclusive due to task variation, a complex
sequential task was administered to individuals with, without, and recovering
from a major depressive episode. Generally, depressed individuals were more
likely to decide rationally compared to non-depressed and recovering patients.
\categorization{P}

  \item[\textcite{DeVries2012}] When no trade-offs are present and a logical rule
can be derived for a dominant choice, individuals in positive moods tend to
make less logical and worse choices. Conversely, negative moods lead
individuals to stick to logical rule-based decision strategies even after
experiencing bad outcomes repeatedly.  \categorization{P, E}

  \item[\textcite{Morgan2013}] Risk-taking might be decreased by eliciting positiv
moods over negative moods in the workplace. In particular, certain moods
directly and indirectly influence risky decisions related to safety-critical
work. \categorization{P, E}

  \item[\textcite{Vinckier2018}] The ventrolateral prefrontal cortex (vmPFC) and anterior
insula (aIns) are correlated to changes to mood levels based on fMRI validation of
a neuro-computational model. At onset of choices, higher vmPFC activity
predicts overweighting of potential gains, while higher aIns activity predicts
overweighting of potential losses. \categorization{N}

\end{description}

\section*{Proposals}

\begin{description}

\item[1] Numerous studies point to the decreased risky behaviors observed in
individuals and groups with negative mood states
\autocite{Yuen2003,VanKnippenberg2010,VonHelversen2011,DeVries2012}. However,
certain domain-specific findings \autocite{Morgan2013} demonstrate positive mood
states to increase risk aversion. Given the domain-specific contradiction to
other evidence, the direction of mood state may be insufficient as an indicator
of willingness to take risks. The particular domain could interact with the
effect of mood state due to the occupations' cognitive requirements. To
investigate this, we might recruit samples of individuals in varied occupations
and randomly assign to conditions of induced positive, neutral, and negative mood states. Between
conditions, subjects play a simple trade-off task with both risky and
conservative options. Analyzing results by occupational group and by mood state
might demonstrate and corroborate interactions like those found in previous domain-specific
work.

\item[2] Depressed individuals tend to fair worse than non-depressed other in a modified Iowa Gambling
Task with large future punishments and large future rewards \autocite{Must2006}.
However, medications for depression may alter the behaviors expected from
non-medicated depressed individuals. Since certain emotion regulation strategies tend
to lead to more rational decisions, a cognitive-behavioral approach to
depression treatment might alter decision-making behaviors of depressed
individuals relative to non-depressed individuals. This could be examined by
administering a similar modified Iowa Gambling Task as the aforementioned study
to depressed individuals undergoing cogntive-behavioral therapies, medication
therapies, or both medication and cognitive-behavioral therapies. Analysis might demonstrate
whether a multi-modality treatment method produces improved decision-making
than those undergoing only one.

\end{description}

\printbibliography[
  title={References}
]

\end{document}

